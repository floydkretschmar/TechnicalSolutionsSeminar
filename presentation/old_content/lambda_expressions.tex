\begin{frame}[fragile]{Lambda-Ausdrücke}{Funktions-Interfaces}

    \centering
    \begin{minipage}[b]{0.75\textwidth}
        \begin{block}{\citet{goetz13}}
            \enquote{So, we have instead followed the path of \textbf{"use what you know"} -- since existing libraries use functional interfaces
            extensively, we codify and leverage this pattern.}
        \end{block}
    \end{minipage}

    \begin{itemize}
        \item<2-> \textbf{functional interfaces}: Interfaces mit genau einer Methode
        \item<3-> \texttt{@FunctionalInterface}-Annotation kann verwendet werden um 
        Designintention zu verdeutlichen
        \item<4-> Eine Vielzahl vordefinierter Interfaces dieser Art existieren in Java-8:\\
        \texttt{Consumer<T>}, \texttt{Function<T,R>}, \texttt{UnaryOperator<T>}, \dots
    \end{itemize}
\end{frame}


\begin{frame}[fragile]{Anonyme innere Klassen}{Nachteile}
    \begin{center}
        \begin{minipage}[b]{0.75\textwidth}
            \begin{block}{Instanziierung eines ActionListeners \citep{goetz13}}
                \begin{onlyenv}<1>
                    \begin{lstlisting}
button.addActionListener(new ActionListener() {
    public void actionPerformed(ActionEvent e) {
        ui.dazzle(e.getModifiers());
    }
});
                    \end{lstlisting}
                \end{onlyenv}
                \begin{onlyenv}<2->
                    \begin{lstlisting}
button.addActionListener(
    %\textcolor{dkred!70}{e -> ui.dazzle(e.getModifiers())}%
);
                    \end{lstlisting}
                \end{onlyenv}
            \end{block}
        \end{minipage}
    \end{center}

    \begin{itemize}
        \item \textcolor<3->{dkgreen!70}{Unhandliche Syntax}
        \item \textcolor<4->{dkgreen!70}{Semantik des Klassenladens und der Objektinstanziierung}
    \end{itemize}

\end{frame}

\begin{frame}[fragile]{Lambda-Ausdrücke}{Alternative Syntax für anonyme Klassen}

    \begin{center}
        \begin{minipage}[b]{0.75\textwidth}
            \begin{block}{Allgemeine Form eines Lambda-Ausdrucks \citep{goetz13}}
                \begin{lstlisting}
(argument1, argument2, ..., argumentN) -> 
{
    ... 
    return ... ;
}
                \end{lstlisting}
            \end{block}
        \end{minipage}
    \end{center}
    
    \begin{itemize}
        \item<2->\textbf{Parameterliste:} Klammern bei weniger als 2 Elementen optional
        \item<3->\textbf{Körper:} 
        \begin{itemize}
            \item<3-> \texttt{break} und \texttt{continue} sind auf oberster Ebene verboten
            \item<4-> jeder Pfad musst etwas zurückgeben oder eine Exception werfen
            \item<5-> bei einzeiligem Körper sind Klammern und \texttt{return} optional
        \end{itemize}
    \end{itemize}
\end{frame}


