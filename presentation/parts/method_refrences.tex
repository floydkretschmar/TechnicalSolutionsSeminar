\begin{frame}[fragile]{Referenzierung von Methoden}

    \begin{itemize}
        \item<1-> Lambda-Ausdrücke vereinfachen die anonyme Instanziierung von funktionalen 
        Interfaces
        \item<2-> \textbf{Ziel:} definiere vereinfachte Form um bereits existierende 
        Methoden zu referenzieren 
    \end{itemize}
    
    \pause

    \begin{center}
        \begin{minipage}[b]{0.75\textwidth}
            \begin{block}{Methoden-Referenzen \citep{goetz13}}
                \begin{lstlisting}
Comparator<Person> byName 
    = Comparator.comparing(%\only<2>{p -> p.getName()}\only<3->{\textcolor{dkred!70}{Person::getName}}%);
Arrays.sort(people, byName);
                \end{lstlisting}
            \end{block}
        \end{minipage}
    \end{center}
\end{frame}

\begin{frame}[fragile]{Methoden-Referenzen}{Parametertypen}
    \begin{itemize}
        \item<1-> Parametertypen der Methode des funktionalen Interfaces fungieren als Argumente
        eines impliziten Methodenaufrufs
        \item<2-> Manipulation der Parametertypen durch widening, boxing, etc. erlaubt
    \end{itemize}

    \pause

    \begin{center}
        \begin{minipage}[b]{0.75\textwidth}
            \begin{block}{Methoden-Referenzen \citep{goetz13}}
                \begin{lstlisting}
// void exit(int status)
Consumer<Integer> b1 = System::exit;
// void sort(Object[] a)
Consumer<String[]> b2 = Arrays::sort;  
// void main(String... args) 
Consumer<String> b3 = MyProgram::main;  
// void main(String... args)
Runnable r = MyProgram::main;           
                \end{lstlisting}
            \end{block}
        \end{minipage}
    \end{center}
\end{frame}

\begin{frame}[fragile]{Methoden-Referenzen}{Methoden-Typen}
    \only<1>{Unterschiedliche Methoden-Typen werden auf verschiedene Arten referenziert:}

    \begin{onlyenv}<2->
        \begin{center}
            \begin{minipage}[b]{0.75\textwidth}
                \begin{block}{Statische Methode \citep{goetz13}}
                    \begin{lstlisting}
KlassenName::methodenName
                    \end{lstlisting}
                \end{block}
            \end{minipage}
        \end{center}

        \begin{itemize}
            \item Klassenbezeichner steht vor dem Trennzeichen.
        \end{itemize}
    \end{onlyenv}

    \begin{onlyenv}<3->
        \begin{center}
            \begin{minipage}[b]{0.75\textwidth}
                \begin{block}{\texttt{super}-Methode \citep{goetz13}}
                    \begin{lstlisting}
super::methodenName
                    \end{lstlisting}
                \end{block}
            \end{minipage}
        \end{center}

        \begin{itemize}
            \item \texttt{super}-Schlüsselwort steht vor dem Trennzeichen
        \end{itemize}
    \end{onlyenv}
\end{frame}

\begin{frame}[fragile]{Methoden-Referenzen}{Methoden-Typen}
    \begin{center}
        \begin{minipage}[b]{0.75\textwidth}
            \begin{onlyenv}<1-2>
                \begin{block}{Methode eines bestimmten Objekts \citep{goetz13}}
                    \begin{lstlisting}
objektName::methodenName
                    \end{lstlisting}
                \end{block}
            \end{onlyenv}
            \begin{onlyenv}<3-5>
                \begin{block}{Methode eines beliebigen Objekts \citep{goetz13}}
                    \begin{lstlisting}
KlassenName::methodenName
                    \end{lstlisting}
                \end{block}
            \end{onlyenv}
            \begin{onlyenv}<6-7>
                \begin{block}{Klassen-Konstruktor \citep{goetz13}}
                    \begin{lstlisting}
KlassenName::new
                    \end{lstlisting}
                \end{block}
            \end{onlyenv}
            \begin{onlyenv}<8->
                \begin{block}{Array-Konstruktor \citep{goetz13}}
                    \begin{lstlisting}
TypeName[]::new
                    \end{lstlisting}
                \end{block}
            \end{onlyenv}

        \end{minipage}
    \end{center}

    \begin{onlyenv}<1-2>
        \begin{itemize}
            \item<1-> Objektbezeichner steht vor dem Trennzeichen.
            \item<2-> bietet bequeme Art um zwischen verschiedenen funktionalen Interfaces zu konvertieren:
        \end{itemize}

        \begin{onlyenv}<2>
            \begin{center}
                \begin{minipage}[b]{0.75\textwidth}
                    \begin{block}{Interface-Konvertierung \citep{goetz13}}
                        \begin{lstlisting}
Callable<Path> c = ...
PrivilegedAction<Path> a = c::call;    
                        \end{lstlisting}
                    \end{block}
                \end{minipage}
            \end{center}
        \end{onlyenv}
    \end{onlyenv}
    \begin{onlyenv}<3-5>
        \begin{itemize}
            \item<3-> Klasse des beliebigen Objekts steht vor dem Trennzeichen
            \item<4-> Objekt auf dem die Methode ausgeführt wird, ist erster Parameter
        \end{itemize}

        \begin{onlyenv}<5>
            \begin{center}
                \begin{minipage}[b]{0.75\textwidth}
                    \begin{block}{Mehrdeutigkeit mit statischer Methode \citep{Gosling14}}
                        \begin{lstlisting}
class C {
    int size() { return 0; }
    static int size(Object arg) { return 0; }
    void test() { Function<C, Integer> f1 = C::size; }
}
                        \end{lstlisting}
                    \end{block}
                \end{minipage}
            \end{center}
        \end{onlyenv}
    \end{onlyenv}
    \begin{onlyenv}<6-7>
        \begin{itemize}
            \item<6-> Klassenbezeichner steht vor und \texttt{new}-Schlüsselwort nach dem Trennzeichen
            \item<7-> wenn Konstruktor überladen $\rightarrow$ Compiler wählt Konstruktor der beste 
            Übereinstimmung mit der Zieltyp hat.
            \item<7-> Typ von generischen Klassen kann explizit angegeben oder inferiert werden
        \end{itemize}
    \end{onlyenv}
    \begin{onlyenv}<8->
        \begin{itemize}
            \item<8-> Typ-Bezeichner des Arrays steht vor und \texttt{new}-Schlüsselwort nach dem Trennzeichen
            \item<9-> werden behandelt wie Konstruktor mit einem einzelnen \texttt{int}-Parameter 
        \end{itemize}
        \begin{onlyenv}<9->
            \begin{center}
                \begin{minipage}[b]{0.75\textwidth}
                    \begin{block}{Spezieller Array-Konstruktor \citep{goetz13}}
                        \begin{lstlisting}
IntFunction<int[]> arrayMaker = int[]::new;
int[] array = arrayMaker.apply(10);
                        \end{lstlisting}
                    \end{block}
                \end{minipage}
            \end{center}
        \end{onlyenv}
    \end{onlyenv}
\end{frame}