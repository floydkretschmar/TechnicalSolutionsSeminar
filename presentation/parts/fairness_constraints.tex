\begin{frame}{The Machine Learning Framework: How is it unfair? \cite{Barocas.2016}}
    \underline{\textbf{Target Variable or class Labels:}} \\
    \begin{itemize}
        \item target variable defines what data miners are looking for (outcome of interest)
        \item class labels divide all possible values of the target variable into mutually exclusive categories
    \end{itemize}

    \begin{block}{\textbf{Unfairness happening through:}}
    \begin{itemize}
        \item Subjective process (\textbf{e.g.} creditworthiness)
        \item Could be unintentionally be parsed in a way which \textbf{systematically disadvantages} protected classes (\textbf{e.g.} hiring decisions based on predicted tenure than worker productivity)
    \end{itemize}
        \end{block}
\end{frame}

\begin{frame}{The Machine Learning Framework: How is it unfair? \cite{Barocas.2016}}
    \underline{\textbf{Training Data:}} Basis of used algorithms and outcome of the model \newline 

    \begin{block}{\textbf{Unfairness happening through:}}
    \begin{itemize}
        \item Biased training data leads to discriminatory models
        \item 1. Cases in which prejudice has played some role
        \item 2. Biased sample of the population (under-/overrepresenttation) leads to skewed results\newline
        $\rightarrow$ less \textbf{accurate}, \textbf{precise} and \textbf{complete} records for certain classes
    \end{itemize}
        \end{block}
\end{frame}

\begin{frame}{The Machine Learning Framework: How is it unfair? \cite{Barocas.2016}}
    \underline{\textbf{Feature Selection:}} Observed and considered attributes \newline 
    
    \begin{block}{\textbf{Unfairness happening through:}}
    \begin{itemize}
        \item Factors that better account for a protected class are not well represented in the set of selected features 
        \item \textbf{e.g.} hiring descisions, enourmous weight to reputation of college (equal competent graduates of protected classes at low rates $\rightarrow$ individuals get discriminnated)
    \end{itemize}
        \end{block}
\end{frame}

\begin{frame}{The Machine Learning Framework: How is it unfair? \cite{Barocas.2016}}
    \underline{\textbf{Proxies:}} Non sensitive features correlate to class membership
    \begin{itemize}
        \item Do not artificially introduce discriminatory effects into the data mining process
        \item Can still result in systematically less favorable determinations for members of protected classes
    \end{itemize} 

    \begin{block}{\textbf{Unfairness happening through:}}
    \begin{itemize}
        \item  Redundant Encoding
        \item \textbf{e.g.} same criteria \textbf{correctly sorts} individuals according to their \textbf{predicted likelihood of excelling at a job} also sort individuals according to \textbf{class membership}
    \end{itemize}
        \end{block}
\end{frame}

\begin{frame}{The Machine Learning Framework: How is it unfair? \cite{Barocas.2016}}
    \underline{\textbf{Masking:}} Decision makers mask their prejudicial views by using the mentioned methods \newline 
    
    \begin{block}{\textbf{Unfairness happening through:}}
    \begin{itemize}
        \item Bias the collection
        \item Preserve the known effects of prejudice in prior decision making
        \item Use prejudical features and/or proxies
    \end{itemize}
        \end{block}
\end{frame}