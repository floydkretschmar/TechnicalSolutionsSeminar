In this section you can summarize and link your assigned reading. Try to distill an overall insight 
from the papers, not to make a laundry list of individual results. Did you come across open questions 
that were not answered in the papers? Are there hidden pitfalls or problems that, in your opinion, the 
papers do not solve or marginalize? Provide a critical but constructive reading without being dismissive. 
Ideally, try to do some literature research of your own to find follow-on papers or related works. 

Include additional work like: \cite{automatedDsicrimination} \cite{Singh} \cite{DBLP:conf/maics/RalescuR17}

The authors of \cite{Berk.2018} argue that the discussion about fairness in machine learning 
is fundamentally a discussion about tradeoffs. They state that \enquote{\dots there will 
always be tradeoffs. These are mathematical facts subject to formal proofs. Denying that these
tradeoffs exist is not a solution.} \\
They argue that the problem at the heart of fairness in machine learning is the difference
in base rate across protected groups which \enquote{can cascade through fairness assessments
and lead to difficult tradoffs.} In terms of discussing these tradeoffs, the authors make multiple
suggestions:
\begin{itemize}
    \item The tradeoffs need to be explicitly represented and available as tuning parameters.
    \item Future measures of fairness should be formulated in such a way, that tradeoffs can
    be made with them.
    \item The determination of tradeoffs should ultimately fall in the hands of the stakeholders.
\end{itemize}
As a final point the paper states, that any solution will likely not come fast and no singular
solution will be able to reverse longstanding, institutionalized inequality.