The rise of machine learning in today's society through automated, data-powered decision making systems in various application fields raised increasing attention regarding the possible inherited discrimination for people with sensitive characteristics (e.g. gender, race or religion) through the commonly used machine learning techniques.
In this context, defining any kind of target variable (outcome of interest) or multiple class labels is always a very subjective process, where 
the presented problem could unintentionally be parsed in a way which systematically disadvantages 
certain classes \cite{Barocas.2016, barocas-hardt-narayanan}. Moreover, the training data could be biased by 
either considering cases in which prejudice has played a role or simply over- or underrepresenting 
a certain class \cite{Barocas.2016, barocas-hardt-narayanan}. If such data gets used, it would lead to a discriminatory model, which would have a unadvantageous impact on a certain subgroup of people. This process can get enhanced through disadvantageous feature selection, proxies (relevant features correlate to class membership) or masking, which allows decision maskers to mask their prejudicial views by using any of the previously mentioned approaches \cite{Barocas.2016, barocas-hardt-narayanan}.  

For the remainder of this paper we want to introduce the statistical framework introduced in \cite{Berk.2018} to describe the process of machine learning. Here Berk et al. propose the idea of a population with a limitless number of i.i.d observations that are sampled from a single joint probability distribution $P(Y,L,S)$. 
The variable $Y$ is the outcome of interest, $L$ represent the legitimate predictors and $S$ are the 
protected predictors like race or gender. In this population there exists a function $f(L,S)$ 
linking the predictors $L,S$ to the expectation of $Y$. When a fitting procedure $h(L,S)$ 
is applied to the data, it produces a so called hypothesis $\hat{f}(L,S)$ which is the 
source of the predictions $\hat{Y}$. The procedure $h(L,S)$ can either be seen as 
approximating the true response surface, resulting in a $\hat{f}(L,S)$ that will be a biased 
estimator for $f(L,S)$. If the estimation target of $h(L,S)$ is instead acknowledged to be 
an approximation of the true response surface, this approximation can be estimated by 
$\hat{f}(L,S)$ in an asymptotically unbiased manner.