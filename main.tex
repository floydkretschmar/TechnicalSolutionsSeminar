\documentclass{article}

% if you need to pass options to natbib, use, e.g.:
\PassOptionsToPackage{numbers, compress}{natbib}
% before loading neurips_2019

% ready for submission
% \usepackage{neurips_2019}

% to compile a preprint version, e.g., for submission to arXiv, add add the
% [preprint] option:
%     \usepackage[preprint]{neurips_2019}

% to compile a camera-ready version, add the [final] option, e.g.:
 \usepackage[final]{neurips_2019}

% to avoid loading the natbib package, add option nonatbib:
%     \usepackage[nonatbib]{neurips_2019}

\usepackage[utf8]{inputenc} % allow utf-8 input
\usepackage[T1]{fontenc}    % use 8-bit T1 fonts
\usepackage{hyperref}       % hyperlinks
\usepackage{url}            % simple URL typesetting
\usepackage{booktabs}       % professional-quality tables
\usepackage{amsfonts}       % blackboard math symbols
\usepackage{nicefrac}       % compact symbols for 1/2, etc.
\usepackage{microtype}      % microtypography
\usepackage{csquotes}       % simple quotes
\usepackage{makecell}       % better table formatting

\title{The Societal Challenge: \\ Legal Perspectives on Discrimination}

% The \author macro works with any number of authors. There are two commands
% used to separate the names and addresses of multiple authors: \And and \AND.
%
% Using \And between authors leaves it to LaTeX to determine where to break the
% lines. Using \AND forces a line break at that point. So, if LaTeX puts 3 of 4
% authors names on the first line, and the last on the second line, try using
% \AND instead of \And before the third author name.

\author{%
  Robin Schmidt \& Floyd Kretschmar\\
  MSc Informatik \\
  University of Tübingen\\
  Matriculation number 4255055 and 4205979\\
  \texttt{[rob.schmidt|Marcel.Mustermann]@student.uni-tuebingen.de}
  % Coauthor \\
  % Affiliation \\
  % Address \\
  % \texttt{email} \
}

\begin{document}

\maketitle

\begin{abstract}
  This template provides guidance on the structure for your 4 page report. Please feel free to deviate from the proposed structure if you feel that it is useful; but try to follow the spirit of the guidelines.
  In the abstract, summarize the topic of your report in about 5-8 lines of text. Do not cite your assigned papers here, but instead give a very concise overview over the insights you report on in this text.
\end{abstract}

\section{Introduction}
\label{sec:introduction}
In the context of machine learning defining any kind of target variable (outcome of interest) or class labels is always a very subjective process, where 
the presented problem could unintentionally be parsed in a way which systematically disadvantages 
certain classes \cite{Barocas.2016}. In addition to that, the training data could be biased by 
either considering cases in which prejudice has played a role or simply over- or underrepresenting 
a certain class \cite{Barocas.2016}. If such data gets used, it would lead to a discriminatory model, which would have a unadvantageous impact on certain subgroup of people. (include feature selection, proxies, masking here)

As the statistical framework to describe the process of machine learning \cite{Berk.2018} 
proposes the idea of a population with a limitless number of I.I.D observations that are 
sampled from a single joint probability distribution $P(Y,L,S)$. 
$Y$ is the outcome of interest, $L$ represent the legitimate predictors and $S$ are the 
protected predictors like race or gender. In this population there exists a function $f(L,S)$ 
linking the predictors $L,S$ to the expectation of $Y$. When a fitting procedure $h(L,S)$ 
is applied to the data, it produces a so called hypothesis $\hat{f}(L,S)$ which is the 
source of the predictions $\hat{Y}$. The procedure $h(L,S)$ can either be seen as 
approximating the true response surface, resulting in a $\hat{f}(L,S)$ that will be a biased 
estimator for $f(L,S)$. If the estimation target of $h(L,S)$ is instead acknowledged to be 
an approximation of the true response surface, this approximation can be estimated by 
$\hat{f}(L,S)$ in an asymptotically unbiased manner.
\section{Relevant Work}

\subsection{Fairness and machine learning: Limitations and Opportunities}
Discuss your first assigned paper \cite{barocas-hardt-narayanan}. Outline the main idea and key results. If suitable, reproduce key mathematical insights. Ideally, also provide critical comments of your own were suitable. But make sure to clearly delineate the ideas and experiments in the assigned paper from your personal opinion or analysis.


\subsection{Fairness in Criminal Justice Risk Assessments: The State of the Art}
In the paper \enquote{Fairness in Criminal Justice Risk Assessments} \cite{Berk.2018} 
the authors explore the concept of fairness in the technical and mathematical context. 
More specifically they are interested to find a way, a notion of fairness can be 
operationalized in the context of machine learning and how it applies to the specific
case of criminal justice risk assessment. The following discussions of fairness are based 
on the statistical framework presented in section \ref{sec:introduction}. For ease of 
exposition the paper limits itself to the case where $Y$ and $\hat{Y}$ are binary.

%Their work can be roughly subdivided into four major parts. In the first part the authors 
%introduce the statistical framework as well as the fundamental mathematical terminology 
%from which they try to derive their notions of fairness. In the second part of the paper
%they go on to define six types of algorithmic fairness, based on the statistical foundations 
%laid out before. Then follows a discussion of the tradeoffs that have to be made when trying
%to achieve different notions of fairness. Both the tradeoff with regards to the overall
%prediction accuracy as well as the tradeoff between different kinds of fairness is 
%discussed. Finally the authors introduce a range of potential solutions to enforce fairness
%in machine learning algorithms.

In the first parts of the paper, the underlying statistical framework is defined. The 
fairness definitions proposed are based on the accuracy measurements defined by the 
confusion matrix \footnote{ A full explanation of the confusion matrix can be found in 
section \ref{sec:appendix}}. More specifically the authors define fairness as an equality 
of accuracy across all protected group categories. Imposing this equality constraint for 
each kind of accuracy defined by the confusion matrix, leads to the following definitions 
of fairness: 

\begin{enumerate}
    \item \textbf{Overall accuracy equality:} equal probability of correct classification 
    $\frac{t_p + t_n}{N}$ 
    \item \textbf{Statistical parity:} equal probability of predicting failure/success 
    $\frac{t_p + f_p}{N}$ or $\frac{t_n + f_n}{N}$
    \item \textbf{Conditional procedure accuracy equality:} equal probability of correct 
    classification, given the actual outcome: $\frac{t_p}{t_p + f_n}$ or 
    $\frac{t_n}{t_n + f_p}$
    \item \textbf{Conditional use accuracy equality:} equal probability of an actual 
    outcome, given the prediction: $\frac{t_p}{t_p + f_p}$ or $\frac{t_n}{t_n + f_n}$
    \item \textbf{Treatment equality:} equal ratio between false negatives and positives: 
    $\frac{f_p}{f_n}$ and $\frac{f_n}{f_p}$
    \item \textbf{Total fairness:} All previously notions of fairness are achieved 
    simultaneously
\end{enumerate}

%\begin{itemize}
%    \item \textbf{Base Rate:} probability of actual failures/successes $\frac{t_p + f_n}{N}$ and 
%    $\frac{t_n + f_p}{N}$
%    \item \textbf{Prediction Distribution:} probability of predicted failures/successes 
%    $\frac{t_p + f_p}{N}$ and $\frac{t_n + f_n}{N}$
%    \item \textbf{Overall Procedure Accuracy:} probability of correct classification: 
%    $\frac{t_p + t_n}{N}$
%    \item \textbf{Conditional Procedure Accuracy:} probability of correct classification, given 
%    the actual outcome: $\frac{t_p}{t_p + f_n}$ and $\frac{t_n}{t_n + f_p}$
%    \item \textbf{Conditional Use Accuracy:} probability of an actual outcome, given the 
%    prediction: $\frac{t_p}{t_p + f_p}$ and $\frac{t_n}{t_n + f_n}$
%    \item \textbf{Cost Ratio:} ratio between false negatives and positives: $\frac{f_p}{f_n}$ and 
%    $\frac{f_n}{f_p}$
%\end{itemize}

For the discussion of fairness it is assumed that $f(L,S) = \hat{f}(L,S)$ as to no 
conflate discussions about fairness and accuracy. But the notions about accuracy discussed 
in section \ref{sec:introduction} also hold true for the probabilities defined by the
confusion matrix of $Y$ and $\hat{Y}$. 

In the next part follows a discussion of necessary tradeoffs between accuracy and 
fairness as well as between different kind of fairness. The paper states that \enquote{
\dots excluding $S$ will reduce accuracy. Any procedure that even just discounts the role 
of $S$ will lead to less accuracy.}\cite{Berk.2018} The authors also explore the conflict 
between conditional use and procedure accuracy equality by citing the following impossibility 
theorem: \enquote{When the base rates\footnote{ proportion of actual failures/successes 
$\frac{t_p + f_n}{N}$ or $\frac{t_n + f_p}{N}$} differ by protected group and when there 
is not separation\footnote{ separation = \enquote{perfectly accurate classification is
possible}\cite{Berk.2018}}, one cannot have both conditional use accuracy and equality in 
the false negative and false positive rates.}\cite{DBLP:journals/corr/KleinbergMR16}
\cite{Chouldechova2017FairPW}. The authors suggest, that \enquote{the key tradeoff will be 
between the false positive and false negative rates on the one hand and the conditional use 
accuracy on the other.}\cite{Berk.2018}

In the final parts of the paper multiple approaches for solving the issue of fairness in 
machine learning are introduced and briefly discussed. The paper explores three different
main strategies, which can be combined.
\begin{enumerate}
    \item \textbf{Pre-Processing} is the elimination of sources of unfairness in the data before 
    formulating $h(L,S)$. Examples include the removal of linear dependencies between $L$ and
    $S$, the rebalancing of base rates or the random transformation of predictors, such that
    $P(Y,L,S)$ is less dependent on $S$.
    \item \textbf{In-Processing} means including the adjustments for fairness in the process 
    of constructing $h(L,S)$. One example of this, is enforcing more fair results according 
    to the defined notions of fairness, if the initial prediction $\hat{Y}$ had substantial 
    uncertainty.
    \item In \textbf{Post-Processing} $h(L,S)$ is applied first, and its results are adjusted
    afterwards to account for fairness. A possible approach for Post-Processing is the 
    reassignment of class labels after classification with the goal of minimizing 
    classification errors subject to a particular fairness constraint.
\end{enumerate}

The authors note, that all the corrections for fairness presented are themselves agnostic 
about \enquote{what the target outcome for fairness should be.}\cite{Berk.2018} They 
argue that a discussion about the benchmark according to which equality is achieved, is just as
important, as it makes the determination of tradeoffs more complicated.

\subsection{Big Data’s Disparate Impact}
The whitepaper "Big Data’s Disparate Impact" \cite{Barocas.2016} by Solon Barocas and 
Andrew D. Selbst is separated into three main parts, which deal with slightly different 
topics regarding fairness in machine learning, in particular data mining. The first part 
focuses on the various ways through which data mining can discriminitate certain classes, 
while the second and third part discuss the liability issue of discrimination in data 
mining for the american title VII (equal employment opportunity) \cite{titleVII} of the 
civil rights act and the difficulty for future legal reforms.  According to their studies, there are five main structures in data mining which can 
cause discrimination for certain classes. In particular, these are the "definition of 
the target variable and class labels" (I), "training data" (II), "feature selection" (III), 
"proxies" (IV) and "masking" (V) \cite{Barocas.2016}. All of these topics have already 
been clarified and described regarding their extent in section \ref{sec:introduction} and 
therefore won't need special attention here.

In the american civil rights act, especially in title VII, there are two presented cases 
for discrimination, namely "disparate treatment" and "disparate impact", which also find 
usage in the presented whitepaper. While disparate treatment describes an uneven behavior 
towards a certain person due to a particular characteristic (e.g. gender, race or 
religion), disparate impact represents a neutral rule which treats everyone equally in 
form, but has a damaging effect on a subset of people with such a certain characteristic \cite{titleVII}.

In their whitepaper Barocas and Selbst argue that formal liability in disparate treatment 
doesn't correspond to any special step within data mining and that using a protected 
class as an input for any classificatory model should be a legal harm, because this 
process corresponds to the employer classifying and differentiating potential hires 
according to exactly this protected class \cite{Barocas.2016}. They also show that the 
disparate treatment either occurs at the decision to apply  a biased predictive model 
or when the biased result gets used for the ultimate hiring decision and draw the 
conclusion that the disparate treatment doctrine doesn't regulate discriminatory 
data mining to a satisfying extent \cite{Barocas.2016}. 

While considering the disparate impact doctrine, the authors state that in such a case 
the plaintiff must prove that "a particular facially neutral employment practice causes 
a disparate impact with respect to a protected class" \cite{Barocas.2016} 
\cite{titleVII}.\footnote{ 42 U.S.C.§2000e-2(k)(1)(A) } In response, the defendant-employer 
is then allowed to justify the challanged practice by showing the job relation and 
business necessity.\footnote{ \textit{Id.} } The plantiff then still has the chance 
to show that an alternative, less discriminatory employment practice could have been 
used instead \cite{Barocas.2016}. For the case of data mining this means that liability 
regarding disparate impact can be caused by using a non job related target variable 
\cite{Barocas.2016}. As soon as the target variable is shown to be job related, there 
are two questions which need to be answered. First, whether or not the model is 
predictive of the trait and secondly if the model with statistical significance 
predicts what it is supposed to predict \cite{Barocas.2016}. Barocas and Selbst 
also explain that it is hard to know which features would make an existing model more
or less discriminatory and therefore proving that a less discriminatory alternative 
would exist becomes a very hard task to solve \cite{Barocas.2016}.

The presented \textit{internal} issues with data mining are fundamental questions that need to be addressed or can't be solved properly. For example, the target variable will always inherent certain kinds of biases, because a target variable must contain judgments about what is really important in the presented problem \cite{Barocas.2016}. Additionally, a solution to the issues with training data labeling need to compromise between forbidding employers from using past discrimination and allowing them to use historical data of good employees \cite{Barocas.2016}. For skewed data sets the employer needs to recognize the type of bias, have access to the underlying data and needs the possibility to collect more data  \cite{Barocas.2016}. Otherwise, oversampling underrepresented communities can clear up some of the bias \cite{Barocas.2016}. Statistical discrimination in the area of feature selection is avoidable if there is the possibility to gather additional or more granular data \cite{Barocas.2016}. Otherwise minimizing the error rate between groups can help to improve this aspect \cite{Barocas.2016}. Lastly, for proxies there needs to be a threshold which defines when a correlation between an attribute and class membership becomes worrisome, as well as when it is sufficiently relevant despite being highly correlated to class membership \cite{Barocas.2016}.



In the american civil rights act, especially in title VII, there are two presented cases for discrimination, namely "disparate treatment" and "disparate impact", which also find usage in the presented whitepaper. While disparate treatment describes an uneven behavior towards a certain person due to a particular characteristic (e.g. gender, race or religion), disparate impact represents a neutral rule which treats everyone equally in form, but has a damaging effect on a subset of people with such a certain characteristic.

In their whitepaper Barocas and Selbst argue that formal liability in disparate treatment doesn't correspond to any special step within data mining and that using a protected class as an input for any classificatory model should be a legal harm, because this process corresponds to the employer classifying and differentiating potential hires according to exactly this protected class \cite{Barocas.2016}. They also show that the disparate treatment either occurs at the decision to apply  a biased predictive model or when the biased result gets used for the ultimate hiring decision and draw the conclusion that the disparate treatment doctrine doesn't regulate discriminatory data mining to a satisfying extent \cite{Barocas.2016}. 

While considering the disparate impact doctrine, the authors state that in such a case the plaintiff must prove that "a particular facially neutral employment practice causes a disparate impact with respect to a protected class" \cite{Barocas.2016} \cite{titleVII}.\footnote{ 42 U.S.C.§2000e-2(k)(1)(A) } In response, the defendant-employer is then allowed to justify the challanged practice by showing the job relation and business necessity.\footnote{ \textit{Id.} } The plantiff then still has the chance to show that an alternative, less discriminatory employment practice could have been used instead \cite{Barocas.2016}. For the case of data mining this means that liability regarding disparate impact can be caused by using a non job related target variable \cite{Barocas.2016}. As soon as the target variable is shown to be job related, there are two questions which need to be answered. First, whether or not the model is predictive of the trait and secondly if the model with statistical significance predicts what it is supposed to predict \cite{Barocas.2016}. Barocas and Selbst also explain that it is hard to know which features would make an existing model more or less discriminatory and therefore proving that a less discriminatory alternative would exist becomes a very hard task to solve \cite{Barocas.2016}.

The presented difficulty for reform possibilities can be separated into \textit{internal} data mining issues as well as \textit{external} politicial and constitutional  constraints \cite{Barocas.2016}. 

\subsection{Machine Bias}
The article \enquote{Machine Bias} by Julia Angwin and Jeff Larson describes the findings 
of ProPublica with regards to the risk assessment tool COMPAS. Their findings seem to confirm
a lot of the problems described by \cite{Barocas.2016}, \cite{barocas-hardt-narayanan} and
\cite{Berk.2018}. First of all, they found problems with regards to overall accuracy of 
predicting future crimes, which was \enquote{only 61 percent [...] for [committing] any 
subsequent crimes withing two years.} \cite{machinebias}. Moreover they found that false 
positive rates for black defendants was \enquote{almost twice the rate as white defendants}
\cite{machinebias} and \enquote{White defendants were mislabled as low risk more often 
than black defendants} \cite{machinebias}.

The authors go on to explain history of criminal risk assessment in the US overall and 
explain the motivation for deploying algorithmic risk assessment: \enquote{If computers
could accurately predict [...] new crimes, the criminal justice system could be fairer 
and more selective about who is incarcerated and for how long.} \cite{machinebias}. The 
article then explores in more detail, how modern algorithmic tools have performed in 
making risk predictions and how these predictions differ among protected groups.

In the final part of the article the authors describe how algorithmically generated risk 
scores generated are being used today. They explain how these systems are deployed in 
various US states and jurisdictions and which degree of impact they have had on decision 
making processes of judges, prosecutes, defenders and other parts of the criminal justice 
system.

\section{Discussion}
From the presented papers as well as additional literature research for the missing EU law structures we conclude that currently the discrimination law structures aren't well prepared for the challenges which are brought up by automated decision making systems \cite{Barocas.2016, barocas-hardt-narayanan, automatedDsicrimination}. This applies for article 14 of the European Convention on Human Rights \cite{EU14}, the article 21 of the Charter of Fundamental rights of the EU \cite{EU21}, as well as title VII \cite{titleVII} of the american civil rights act  \cite{Barocas.2016, automatedDsicrimination}. Moreover, it is going to be interesting how the European General Data Protection Regulation (GDPR) will influence this subject in the upcoming years \cite{automatedDsicrimination, Singh}.

All of the presented papers describe issues within the machine learning process which are hard if not impossible to solve for future liability improvements and the authors of \cite{ Barocas.2016, Berk.2018} argue that discussion about fairness in machine learning is fundamentally a discussion about tradeoffs. Berk et al. state that the problem at the heart of fairness in machine learning is the difference
in base rate across protected groups which \enquote{can cascade through fairness assessments
and lead to difficult tradoffs.} \cite{Berk.2018}. In terms of discussing these tradeoffs, the authors make multiple suggestions. First, the tradeoffs need to be explicitly represented and available as tuning parameters. Secondly, future measures of fairness should be formulated in such a way, that trade-offs can be made with them. And thirdly, the determination of tradeoffs should ultimately fall in the hands of the stakeholders.

In this field there have also been various works like \cite{DBLP:conf/kdd/FeldmanFMSV15, isabel02, isabel01} and many more for detecting and removing discrimination in automated decision making systems. 

\section{Summary}
In conclusion, all of the papers present the issue of fairness in machine
learning as a fairly complex one. Recognizing and exploiting patterns in 
data is at the very core of machine learning. If the underlying structures contain
discriminatory or biased patterns, this will be reflected in the resulting
algorithms. The goal has to be to find ways to quantify and operationalize
fairness in a way that makes these biases obvious and allows to account for
them. This becomes especially important, as these systems are used more
and more frequently to make decisions about peoples lives and that have long
lasting effects, like automated decision making systems used in the 
criminal justice system.

\section{Appendix: Possible Additions}
Here we can provide all relevant additional information.
\label{sec:appendix}

\begin{table}[h]
    \caption{Confusion Matrix}
    \label{tab:confusion}
    \centering
    \begin{tabular}{ c|c|c|c } 
    \toprule
      & \thead{$\hat{Y}_f$ \\ Failure predicted} & \thead{$\hat{Y}_s$ \\ Success Predicted} & \thead{Conditional Procedure \\ Accuracy} \\ 
    \hline
    \thead{$Y_f$ \\ Failure - A Positive} & \makecell{$t_p$ \\ true positive} & \makecell{$f_n$ \\ false negative} & \makecell{$\frac{t_p}{t_p + f_n}$ \\ True Positive Rate} \\ 
    \hline
    \thead{$Y_s$ \\ Success - A Negative} & $f_p$: false positive & $t_n$: true negative & \makecell{$\frac{t_n}{t_n + f_p}$ \\ True Negative Rate } \\ 
    \hline
    \thead{Conditional \\ Use Accuracy} & $\frac{t_p}{t_p + f_p}$ & $\frac{t_n}{t_n + f_n}$ & $\frac{t_p + t_n}{t_p + f_p + t_n + f_n}$ \\ 
    \bottomrule
    \end{tabular}
\end{table}

\begin{itemize}
    \item \textbf{Sample Size:} $N = t_p + f_p + t_n + f_n$
    \item \textbf{Base Rate:} proportion of actual failures/successes $\frac{t_p + f_n}{N}$ or $\frac{t_n + f_p}{N}$
    \item \textbf{Prediction Distribution:} proportion of predicted failures/successes $\frac{t_p + f_p}{N}$ or $\frac{t_n + f_n}{N}$
    \item \textbf{Cost Ratio:} ratio between false negatives and positives: $\frac{f_p}{f_n}$ or $\frac{f_n}{f_p}$
\end{itemize}

\medskip
\small
\bibliographystyle{abbrvnat}
\bibliography{bibfile}


\end{document}
