\documentclass{article}

% if you need to pass options to natbib, use, e.g.:
\PassOptionsToPackage{numbers, compress}{natbib}
% before loading neurips_2019

% ready for submission
% \usepackage{neurips_2019}

% to compile a preprint version, e.g., for submission to arXiv, add add the
% [preprint] option:
%     \usepackage[preprint]{neurips_2019}

% to compile a camera-ready version, add the [final] option, e.g.:
 \usepackage[final]{neurips_2019}

% to avoid loading the natbib package, add option nonatbib:
%     \usepackage[nonatbib]{neurips_2019}

\usepackage[utf8]{inputenc} % allow utf-8 input
\usepackage[T1]{fontenc}    % use 8-bit T1 fonts
\usepackage{hyperref}       % hyperlinks
\usepackage{url}            % simple URL typesetting
\usepackage{booktabs}       % professional-quality tables
\usepackage{amsfonts}       % blackboard math symbols
\usepackage{nicefrac}       % compact symbols for 1/2, etc.
\usepackage{microtype}      % microtypography

\title{The Societal Challenge: \\ Legal Perspectives on Discrimination}

% The \author macro works with any number of authors. There are two commands
% used to separate the names and addresses of multiple authors: \And and \AND.
%
% Using \And between authors leaves it to LaTeX to determine where to break the
% lines. Using \AND forces a line break at that point. So, if LaTeX puts 3 of 4
% authors names on the first line, and the last on the second line, try using
% \AND instead of \And before the third author name.

\author{%
  Robin Schmidt \& Floyd Kretschmar\\
  MSc Informatik \\
  University of Tübingen\\
  Matriculation number 4255055 and 1234567\\
  \texttt{[rob.schmidt|Marcel.Mustermann]@student.uni-tuebingen.de}
  % Coauthor \\
  % Affiliation \\
  % Address \\
  % \texttt{email} \
}

\begin{document}

\maketitle

\begin{abstract}
  This template provides guidance on the structure for your 4 page report. Please feel free to deviate from the proposed structure if you feel that it is useful; but try to follow the spirit of the guidelines.
  In the abstract, summarize the topic of your report in about 5-8 lines of text. Do not cite your assigned papers here, but instead give a very concise overview over the insights you report on in this text.
\end{abstract}

\section{Introduction} \label{Introduction}

Description of Fairness in ML (target variable/class labels, training data, feature selection, Proxies, Masking)

Defining any kind of target variable or class labels is always a very subjective process, where the presented problem could unintentionally be parsed in a way which systematically disadvantages certain classes \cite{Barocas.2016}. In addition to that, the training data could be biased by either considering cases in which prejudice has played a role or simply over- or underrepresenting a certain class \cite{Barocas.2016}. If this data gets used, it would lead to a discriminatory model. 

\section{Relevant Work}

Here you should discuss your assigned works individually (if you have been assigned to read several chapters in a textbook, make separate subsections for each chapter).

\subsection{Fairness and machine learning: Limitations and Opportunities}

Discuss your first assigned paper \cite{barocas-hardt-narayanan}. Outline the main idea and key results. If suitable, reproduce key mathematical insights. Ideally, also provide critical comments of your own were suitable. But make sure to clearly delineate the ideas and experiments in the assigned paper from your personal opinion or analysis.

\subsection{Fairness in Criminal Justice Risk Assessments: The State of the Art}

Discuss your second assigned paper \cite{Berk.2018}. Outline the main idea and key results. If suitable, reproduce key mathematical insights. Ideally, also provide critical comments of your own were suitable. But make sure to clearly delineate the ideas and experiments in the assigned paper from your personal opinion or analysis.

\subsection{Big Data’s Disparate Impact}
The whitepaper "Big Data’s Disparate Impact" \cite{Barocas.2016} by Solon Barocas and Andrew D. Selbst is separated into three main parts, which deal with slightly different topics regarding fairness in machine learning, in particular data mining. The first part focuses on the various ways through which data mining can discriminitate certain classes, while the second and third part discuss the liability issue of discrimination in data mining for the american title VII (equal employment opportunity) \cite{titleVII} of the civil rights act and the difficulty for future legal reforms.

According to their studies, there are five main structures in data mining which can cause discrimination for certain classes. In particular, these are the "definition of the target variable and class labels" (I), "training data" (II), "feature selection" (III), "proxies" (IV) and "masking" (V) \cite{Barocas.2016}. All of these topics have already been clarified and described regarding their extent in section \ref{Introduction} and therefore won't need special attention here.

In the american civil rights act, especially in title VII, there are two presented cases for discrimination, namely "disparate treatment" and "disparate impact", which also find usage in the presented whitepaper. While disparate treatment describes an uneven behavior towards a certain person due to a particular characteristic (e.g. gender, race or religion), disparate impact represents a neutral rule which treats everyone equally in form, but has a damaging effect on a subset of people with such a certain characteristic. In their whitepaper Barocas and Selbst argue that formal liability in disparate treatment doesn't correspond to any special step within data mining and that using a protected class as an input for a classificatory model should be a legal harm, because this process corresponds to the employer classifying and differentiating potential hires according to exactly this protected class \cite{Barocas.2016}.



\subsection{Machine Bias}

Discuss your forth assigned paper. Outline the main idea and key results. If suitable, reproduce key mathematical insights. Ideally, also provide critical comments of your own were suitable. But make sure to clearly delineate the ideas and experiments in the assigned paper from your personal opinion or analysis.

\section{Discussion}

In this section you can summarize and link your assigned reading. Try to distill an overall insight from the papers, not to make a laundry list of individual results. Did you come across open questions that were not answered in the papers? Are there hidden pitfalls or problems that, in your opinion, the papers do not solve or marginalize? Provide a critical but constructive reading without being dismissive. Ideally, try to do some literature research of your own to find follow-on papers or related works. 

\section{Summary}

Provide a concise summary of your findings, in about 3-10 lines of text.

\section{Appendix: Possible Additions}
Here we can provide all relevant additional information.

\medskip
\small
\bibliographystyle{abbrvnat}
\bibliography{bibfile}


\end{document}
